%\documentclass[12pt]{scrreprt}
\documentclass[12pt]{report}

% language may be romanian or english (default is english)
% type may be bachelor or master (default is bachelor)
\usepackage[language=english, type=bachelor]{style}

%\geometry{a4paper,top=2.5cm,left=3cm,right=2.5cm,bottom=2.5cm}
%in style
%controlling the appearance of your headers and footers
\usepackage{fancyhdr}
\pagestyle{fancy}
\lhead{}
\chead{}
\renewcommand{\headrulewidth}{0.2pt}
\renewcommand{\footrulewidth}{0.2pt}

\begin{document}

\specialization{COMPUTER SCIENCE}
\title{Using artificial intelligence to assist chess players}
\author{Cadar Eduard}
\supervisor{Asist. univ. dr. Florentin Bota}
\maketitle

\specialization{INFORMATICĂ}
\title{Utilizarea inteligenței artificiale în asistarea jucătorilor de șah}
\author{Cadar Eduard}
\supervisor{Asist. univ. dr. Florentin Bota}
\maketitle

\newpage
\thispagestyle{empty}
\mbox{}
\newpage
\pagenumbering{roman} 

\cleardoublepage
ABSTRACT
\vspace{0.5cm}	
\hrule
\vspace{0.5cm}	
%\cleardoublepage

The usual way of searching for a move from a chess position is through tactics or strategy. Tactics are series of moves that would bring an immediate advantage, while strategy refers to a general 'sense' of advantage on the board, without too much calculation (having better developed pieces, better pawn structure etc.).\cite{klein2022neural}

Chess engines are usually built using a minimax algorithm. This approach is able to find the moves that would bring material gain (capturing pieces) in a given depth, but cannot find moves that would slowly improve the position. A neural network trained on professional chess games is able to find good positional moves, but may be weak at finding tactical moves.

An algorithm combining these two approaches should yield better results than each of them on their own.

Abstract: un rezumat \^{i}n limba englez\u{a} cu prezentarea, pe scurt, a con\c{t}inutului pe capitole, pun\^{a}nd accent pe contribu\c{t}iile proprii \c{s}i originalitate

\tableofcontents

\newpage
\pagenumbering{arabic}

\chapter{Introduction}
% \chapter*{Introducere}
\label{intro}

% Introducere: obiectivele lucrarii si descrierea succinta a capitolelor, prezentarea temei, prezentarea contributiei proprii, respectiv a rezultatelor originale

Despite being a centuries-old game, chess still remains unsolved, and is one of the most extensively researched areas of artificial intelligence. Due to the complexity of the game and the large number of possible positions, conventional methods of computing and searching for the best move are ineffective and produce unsatisfactory results, necessitating the use of heuristics in search, evaluation, and move selection.

Chess needs so much creativity and profound reasoning that it was originally considered that computers would never be able to perform it. And it remained that way for a long time, until IBM's Deep Blue \cite{campbell2002deep} defeated reigning World Chess Champion Garry Kasparov in 1997, winning 2 matches, drawing 3, and losing 1. Since then, the best chess engines have improved to the point that no human is able to win a single match against them.

% Chess engines operate ... different than humans - humans look into less branches (are more selective), while engines brute-force almost all the branches

There are several properties of the chess game that make it ideal for computers:
\begin{itemize}
    \item Deterministic - all potential games result in either a win for white, a win for black, or a draw
    \item Finite - games cannot continue indefinitely (there are two restrictions to avoid this: a draw occurs after three repetitions of the same position, and 50 moves without a capture or a pawn move is also a draw)
    \item Complete information - there is no concealed information or ambiguity as there is in a card game, and the game is played sequentially; both players have access to the same information
    \item Zero-sum - the goals of the competitors are opposite: a win for one player is a loss for his opponent, and a draw results in a game sum of zero. If a position is worth +10 to one player, then the opponent's score is -10
\end{itemize}

In the next chapter some of the techniques and algorithms used in programming chess engines are described, as well as some of the optimization methods that modern chess engines utilize. In the third chapter an introduction is given to the state of the art in chess engines and how an engine's strength is determined. In the fourth chapter I will describe the methods and algorithms I used in building my chess engine, and in the fifth chapter I will describe the tools and technologies with which I built the game and the engine. The sixth chapter presents an evaluation of the engine, and in the seventh chapter the conclusions of the work are highlighted.
%\addcontentsline{toc}{chapter}{Introducere}
%\addcontentsline{toc}{chapter}{Introduction}

\chapter{Background}
\label{chap:ch2}

\indent\par Informatii si citare carte \cite{Sommerville2010}.

\section{Methods used}
\label{sec:ch2sec1}

\par Methods/algorithms used in programming and training chess engines.

\section{State of the art chess engines}
\label{sec:ch2sec2}

\par Stockfish, AlphaZero etc. - overview and AI techniques used in them 

% Informatii si citare articol publicat la conferinta (in proceedings) \cite{Narayan2012}.

% Informatii si citare articol publicat în revista \cite{Robbes2015}.

% Informatii si citare articol publicat tip masterthesis  \cite{mastersthesis1993}.

% Informatii si citare articol publicat tip phdthesis \cite{phdthesis1993}.

% Informatii si citare articol ca resursa Internet  \cite{kinaSUR}.

% Inserarea si Referirea unei figuri \ref{FigCBSD}.

% \begin{figure}[htbp]
% 	\centering
% 		\includegraphics[scale=0.65]{./figures/fig_3_1.eps}
% 	\caption{Ciclul de dezvoltare al sistemelor bazate pe componente adaptat modelului cascadã}
% 	\label{FigCBSD}
% \end{figure}

% Inserarea si Referirea la Tabelul \ref{TabelSolutii}.

% \begin{table}[htbp]
% \begin{center}
% \begin{tabular}
% {|p{120pt}|p{120pt}|p{120pt}|}
% \hline
%  Nume algoritm  &  Toate solutiile &  Solutia optimã\\
% \hline 
% \hline Nume 1 & $20$ & $5$  \\
% \hline Nume 2 & $20$ & $2$  \\
% \hline
% \end{tabular}
% \end{center}
% \caption{Solutii obtinute }
% \label{TabelSolutii}
% \end{table}

% Adaugarea si Referirea la o Ecuatie \ref{LabelMyEquation}.

%  \begin{equation}
%      ws_N4 = w_{14}*N1 + W_{24}+N2 + w_{34}*N3
% \label{LabelMyEquation}
%  \end{equation}
 

\chapter{Methodology}
\label{chap:ch3}

Description of the approach taken to build the chess engine\\
Explanation of the AI techniques used and why they were chosen

\section{Training}
\label{sec:ch3sec1}

\par Algorithms/techniques used for training the engine

\subsection{Min-max algorithm}
\label{subsec:ch3sec1subsec1}

\par Used to search for best move to a given depth

\section{Optimizing}
\label{sec:ch3sec2}

\par Algorithms/techniques used for optimizing the engine

\subsection{Alpha-Beta pruning}
\label{subsec:ch3sec2subsec1}

\par Used to detect and cut off branches that will lead to worse results than the ones already analyzed
\chapter{Technologies}
\label{chap:ch4}

Details of the programming languages, libraries, and tools used

\section{Chess game}
\label{sec:ch4sec1}

\par Description of tools used in building the chess game - Unity, C\#

\section{Chess engine}
\label{sec:ch4sec2}

\par Description of tools used in building the chess engine - Python
\chapter{Technologies}
\label{chap:ch5}

% Details of the programming languages, libraries, and tools used

\section{Chess game}
\label{sec:ch5sec1}

% Description of tools used in building the chess game - Unity, C\#

\section{Chess engine}
\label{sec:ch5sec2}

% Description of tools used in building the chess engine - C\#, keras

\chapter{Conclusions}
%\chapter*{Conclusions}
\label{conclusions}

Summary of the main findings and contributions of the thesis\\
Discussion of potential future improvements to the chess engine

%\addcontentsline{toc}{chapter}{Concluzii}
%\addcontentsline{toc}{chapter}{Conclusions}

\bibliography{references}

\end{document}
