\chapter{Technologies and implementation}
\label{chap:ch5}
% Details of the programming languages, libraries, and tools used

\section{Chess engine}
\label{sec:ch5sec1}
% Description of tools used in building the chess engine - C\#, keras

\subsection{C\#}
\label{subsec:ch5sec1subsec1}

The chess engine implementation was written in C\#, as it offers a convenient way of modelling the chess entities through its object-oriented behaviour.

The most preffered alternative for chess engines is C++, because of its execution speed. While C++ might be a better choice in terms of performance \cite{ogala2020comparative}, it is a programming language that is more difficult to develop and maintain due to its complex syntax and low-level memory manipulation. C\# also has robust access to helpful libraries, further simplifying the implementation process.

Another option is Python, which offers greater ease of use, but is generally slower due to the fact that it uses dynamic typing and is an interpreted language, not a compiled one.

\section{Chess game interface}
\label{sec:ch5sec2}
% Description of tools used in building the chess game - Unity, C\#

I created the chess interface in Unity, a game development engine that provides a visual editor and C\# scripting capabilities. I chose Unity because of its user-friendly and intuitive interface, as well as its ability to easily integrate external C\# libraries, such as the chess engine I built.

On the game interface I added inputs that allow the player to change the engine parameters: the minimax depth, the usage of iterative deepening and quiescence search, and also the time limit for the iterative deepening and the maximum depth for the quiescence search.