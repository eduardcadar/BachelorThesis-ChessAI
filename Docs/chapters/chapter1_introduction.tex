\chapter{Introduction}
% \chapter*{Introducere}
\label{intro}

% Introducere: obiectivele lucrarii si descrierea succinta a capitolelor, prezentarea temei, prezentarea contributiei proprii, respectiv a rezultatelor originale

Despite being a centuries-old game, chess still remains unsolved, and is one of the most extensively researched areas of artificial intelligence. Due to the complexity of the game and the large number of possible positions, conventional methods of computing and searching for the best move are ineffective and produce unsatisfactory results, necessitating the use of heuristics in search, evaluation, and move selection.

Chess needs so much creativity and profound reasoning that it was originally considered that computers would never be able to perform it. And it remained that way for a long time, until IBM's Deep Blue \cite{campbell2002deep} defeated reigning World Chess Champion Garry Kasparov in 1997, winning 2 matches, drawing 3, and losing 1. Since then, the best chess engines have improved to the point that no human is able to win a single match against them.

% Chess engines operate ... different than humans - humans look into less branches (are more selective), while engines brute-force almost all the branches

There are several properties of the chess game that make it ideal for computers:
\begin{itemize}
    \item Deterministic - all potential games result in either a win for white, a win for black, or a draw
    \item Finite - games cannot continue indefinitely (there are two restrictions to avoid this: a draw occurs after three repetitions of the same position, and 50 moves without a capture or a pawn move is also a draw)
    \item Complete information - there is no concealed information or ambiguity as there is in a card game, and the game is played sequentially; both players have access to the same information
    \item Zero-sum - the goals of the competitors are opposite: a win for one player is a loss for his opponent, and a draw results in a game sum of zero. If a position is worth +10 to one player, then the opponent's score is -10
\end{itemize}

In the next chapter some of the techniques and algorithms used in programming chess engines are described, as well as some of the optimization methods that modern chess engines utilize. In the third chapter an introduction is given to the state of the art in chess engines and how an engine's strength is determined. In the fourth chapter I will describe the methods and algorithms I used in building my chess engine, and in the fifth chapter I will describe the tools and technologies with which I built the game and the engine. The sixth chapter presents an evaluation of the engine, and in the seventh chapter the conclusions of the work are highlighted.