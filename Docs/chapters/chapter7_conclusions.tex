\chapter{Conclusions}
%\chapter*{Conclusions}
\label{conclusions}

% Summary of the main findings and contributions of the thesis
% Discussion of potential future improvements to the chess engine

Based on the results, the neural network appears to be able to improve the evaluation function slightly, even with a simple network architecture, brief training, and a relatively small dataset, compared to the vast number of positions that can appear in chess.

There are several parts of the engine which could be improved. It is currently very slow, a problem which could be partially solved by implementing a faster move generation algorithm and by representing the chess board as bitboards. Allocating less time to move generation enables the engine to dedicate a larger portion of its available time towards deeper search, allowing it to provide more accurate evaluations.

Another way of improving the engine's speed is through more aggressive pruning, based on some heuristics. This would create the risk of overlooking some moves, but it would help the engine focus on more promising variations.

For a reliable evaluation of the engine's strength, it should implement the UCI (Universal Chess Interface) protocol \cite{uci-protocol} and be uploaded on a site where it receives an ELO rating based on the results of the matches against other engines.